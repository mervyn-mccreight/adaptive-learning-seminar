\chapter{Der LISP Tutor}

%% -- einleitung
%% - wurde an der Carnegie-Mellon-University entwickelt
%% - 1985 entwickelt
%% - soll LISP lehren
%% - idee entstand aus der not, dass viele studenten die lisp-vorlesungen nicht verstanden
%%
%% -- aufbau
%% - benutzerschnittstelle besteht nur aus text und einem textfeld für kommunikation
%% - der generelle ablauf ist:
%% -- system stellt dem benutzer eine aufgabe
%% -- bewertet seine lösung
%% -- gibt für jede eingabe sofort (!) feedback
%% -- stellt daraufhin eine neue aufgabe
%% -- oder gibt hilfestellung
%% -- und so weiter
%%
%% - nun am beispiel durchgehen (komplettes listing 2 schritt für schritt illustrieren):
%% --- abschnitt 1: dialog zwischen student und maschine, um funktion fact zu schreiben (listing 2)
%% -- <aufgabe abschreiben>
%% -- <erste user-ausgabe aufschreiben>
%% -- <erste ausgabe vom programm aufschreiben>
%% --- man sieht, natürlich sprachliche antwort vom programm
%% --- analyse von eingegebenem code, annahme was er bewirken soll
%% --- vorschlag: funktion statt equal zero --> ergebnis von heuristischem wissen.
%%
%% --- listing abschnitt 3
%% -- hier sieht man auch ein beispiel, wo der benutzer eine auswahl hat
%% -- auch in diesen auswahlen gibt es optimale und nicht-optimale lösungen
%% -- schlägt man etwas langfristig unkluges vor, wird in
%% -- textform erklärt warum das der fall ist
%% -- in diesem fall wird erkannt, dass der lernende den rekursions-gedanken
%% -- nicht verstanden hat. nun wird im folgenden mit einem kleinen exkurs
%% -- zum thema rekursion durchgearbeitet, bevor die eigentliche aufgabe
%% -- wieder aufgenommen wird.
%% ---> entscheidung des tutorenmodells auf basis der analyse im lernermodell
%% -- siehe folgedialoge mit theor. fragen zu rekursion.
%%
%% -- und so weiter .. bisschen fleißaufgabe hier
