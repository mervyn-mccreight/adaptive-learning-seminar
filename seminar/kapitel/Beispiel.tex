\chapter{Der LISP-Tutor GREATERP}

\section{Hintergrund}
Ein Beispiel für ein Intelligentes Tutorielles System ist der LISP-Tutor GREATERP\footnote{Goal-Restricted Environment for Tutoring and Educational Research on Programming}.
Dieses Lehrsystem wurde in den 80er Jahren an der Carnegie-Mellon-Universität entwickelt.
Ziel des Lernsystem ist es, einem Lernenden die Prinzipien von LISP zu lehren. Dabei soll das
Gefühl eines auf den Lernenden zugeschnittenen Einzelunterrichts entstehen.\cite[S. 159]{anderson1985}

Das Programm verfügt über eine sehr einfach gehaltene Benutzerschnittstelle.
Die einzige Kommunikationsform zwischen Tutor und Schüler ist Text.
Das Programm präsentiert seine Informationen und Aufgabenstellungen in reiner Textform,
während der Schüler in einem dafür vorgesehenen Editorfeld darauf antworten kann.
Kommt der Schüler an einer Stelle des Programms nicht weiter, hat über das Textfeld auch die Möglichkeit
nach einer genaueren Erklärung zu verlangen.
Da das Lernprogramm vergleichsweise sehr natürlich sprachlich kommuniziert, entsteht der
Eindruck eines echten Dialogs.

Der Ablauf entspricht hierbei dem typischen Ablauf eines ITS.
Das System stellt dem Schüler eine Aufgabe. Der Schüler gibt eine Lösung ab, oder fragt nach genauerer Erklärung.
Die abgegebene Lösung wird von dem System analysiert, sodass sich im Lernermodul ein Modell aktuellen Wissenstands des Schülers bildet.
Das System bewertet im Zusammenspiel des Wissensmodells und des Lernermodells die Lösung und den aktuellen Wissensstand.
Auf Basis dieser Analyse entscheidet das Tutorenmodell, welche Art von Aufgabenstellung sich für den Schüler in der Folge am Besten eignet.
Feedback wird sofort gegeben.\cite[S. 160f]{anderson1985}

\section{Der Beispieldialog - Das Lösen der Fakultät-Aufgabe}
Um die Arbeitsweise des intelligenten LISP-Tutors näher zu erläutern,
wird im Folgenden ein Beispieldialog präsentiert. Dieser ist an der
Carnegie-Mellon-Universität mit Hilfe eines Studenten und dem Lernsystem entstanden.
Die Präsentation der Ausgabe entspricht nicht dem originalen Design des Programms,
sondern aus Gründen der Übersicht tatsächlich als eine Art Dialog dargestellt.
Des weiteren wird der Dialog in verkürzter Form präsentiert, um die wichtigen Aspekte
herauszuarbeiten.
\newpage

%%
%% - nun am beispiel durchgehen (komplettes listing 2 schritt für schritt illustrieren):
%% --- abschnitt 1: dialog zwischen student und maschine, um funktion fact zu schreiben (listing 2)
%% -- <aufgabe abschreiben>
%% -- <erste user-ausgabe aufschreiben>
%% -- <erste ausgabe vom programm aufschreiben>
%% --- man sieht, natürlich sprachliche antwort vom programm
%% --- analyse von eingegebenem code, annahme was er bewirken soll
%% --- vorschlag: funktion statt equal zero --> ergebnis von heuristischem wissen.

\begin{figure}[!ht]
\begin{lstlisting}
  TUTOR:
  Define the function fact. It takes a number argument n, and computes the
  factorial of n, which is the result of multiplying n*(n-1)*(n-2) ... * 1.
  By definition, 0! is 1.

  USER-INPUT:
  (defun fact(n)
    (cond ((equal) < ACTION >)
    <RECURSIVECASE>
  ))

  TUTOR:
  I assume that you are trying to test if the number argument is equal to
  the number zero. There is a special function for testing equivalence to zero.
  Try to use that function instead of equal.

  What code can you write to test if the condition for terminating case is true?
  Try writing a predicate!
\end{lstlisting}
\caption[Beispieldialog LISP-Tutor Teil 1]{Beispieldialog LISP-Tutor Teil 1 \\ Quelle: \cite[S. 173]{anderson1985}}
\end{figure}

Anhand des ersten Teils fällt sofort auf, dass das Lernsystem in Schriftform, aber
auf natürlich sprachliche Art und Weise mit dem Lernenden kommuniziert.
Im zweiten Schritt, nach der Eingabe des ersten Lösungsansatzes, ist zu sehen,
dass das Lernsystem dazu in der Lage ist, den eingegebenen Lösungstext zu verstehen.
Die Lösung wird nicht bloß mit einer Musterlösung verglichen und als falsch tituliert.
Stattdessen erkennt das Programm die Intention und liefert Ansätze zur Verbesserung.
Dies ist ein gutes Beispiel zur Verdeutlichung der Funktionsweise des heuristischen Wissens
im Wissensmodell. Der Fehler, für den Vergleich mit der Zahl null, die Funktion
equal statt zerop zu verwenden, ist ein häufig vorkommender Fehler. Dieser wurde mit Hilfe
des heuristischen Wissens erkannt und durch das Tutorensystem mit einem entsprechenden Tipp
kommentiert.
\newpage


%%
%% --- listing abschnitt 3
%% -- hier sieht man auch ein beispiel, wo der benutzer eine auswahl hat
%% -- auch in diesen auswahlen gibt es optimale und nicht-optimale lösungen
%% -- schlägt man etwas langfristig unkluges vor, wird in
%% -- textform erklärt warum das der fall ist
%% -- in diesem fall wird erkannt, dass der lernende den rekursions-gedanken
%% -- nicht verstanden hat. nun wird im folgenden mit einem kleinen exkurs
%% -- zum thema rekursion durchgearbeitet, bevor die eigentliche aufgabe
%% -- wieder aufgenommen wird.
%% ---> entscheidung des tutorenmodells auf basis der analyse im lernermodell
%% -- siehe folgedialoge mit theor. fragen zu rekursion.

\begin{figure}[!ht]
\begin{lstlisting}
  TUTOR:
  Is the value of fact in the recursive cases calculated by:
  PRESS:      IF YOU WANT TO:
  1           Calling fact recursively with n.
  2           Multiplying n by fact of n minus 1.
  3           Getting 1 2 ... n and then multiplying the results.
  4           Have the tutor choose.

  USER-INPUT:
  3

  TUTOR:
  That is a reasonable way to think of doing factorials, but it is not a plan for a recursive function.
  Since you seem to be having trouble with the recursive cases, let us work through some examples and
  figure out the conditions and actions for each of these cases.
  (...)
\end{lstlisting}
\caption[Beispieldialog LISP-Tutor Teil 2]{Beispieldialog LISP-Tutor Teil 2 \\ Quelle: \cite[S. 173]{anderson1985}}
\end{figure}

Hier sieht man ein Beispiel einer theoretischen Frage, die nicht direkt mit Code beantwortet wird.
In diesen Auswahlen gibt es optimale und nicht-optimale Lösungen.
Diese Frage wird dem Lernenden gestellt, da er in den vorangegangenen Lösungsschritten
bereits Hilfestellungen in Anspruch nehmen musste. Im Lernermodell wird er daher bisher als nicht
fortgeschrittener Benutzer bewertet. Da es sich im Kern um eine Aufgabe zum Thema Rekursion handelt,
wird über die theoretische Zwischenfrage geprüft, ob der Lernende sich bereits mit dem Thema auskennt.
Da die Frage in diesem Beispiel falsch beantwortet wird, werden nun erst die allgemeinen Verständnislücken
zum Thema Rekursion geschlossen, bevor mit der eigentlichen Übungsaufgabe fortgefahren wird.
Hätte der Lernende die Frage richtig beantwortet, wäre der Kurs an dieser Stelle direkt
mit der Weiterführung der Programmierung fortgefahren. Auch hier zeigt sich wieder das Zusammenspiel des Lerner- und des Tutorenmodells.
