\chapter{Einführung}

Die Idee des adaptiven Lernen entstand mit der fortschreitenden
Erforschung des menschlichen Lernprozesses.
Der klassische Frontalunterricht, wie er von früher bekannt ist,
beruht stark auf dem behavioristischen Lernparadigma.
Das Lernen an sich ist ein statischer Prozess, der so angewandt
auf jeden Menschen die selbe Wirkung hat.

Eine modernere Ansicht wird durch den Kognitivismus
beschrieben. Hier rückt der Mensch als Individuum mehr in den Fokus.
Der generelle Ansatz ist hier, dass Lernen nicht mehr eine reine
Ansammlung von Wissenswiedergaben ist. Stattdessen lernt der Mensch
auf individuelle Art und Weise Problemstellungen anhand gewisser Regeln zu lösen.
Der Lernprozess ist hierbei individuell und dynamisch.

Diese Erkenntnisse über das Lernverhalten eines Menschen löste die Diskussion aus,
ob die bisher bekannten Lernmethoden auf die neuen Ansichten angepasst werden sollte.
Das gab den Anstoß zum Erforschen des adaptiven Lernens.
Adaptives Lernen soll diese Individualität berücksichtigen.
Anders als im reinen Frontalunterricht soll beim adaptiven Lernen
der Unterricht dynamisch auf den jeweiligen Lernenden angepasst werden.



Adaptives Lernen im e-Learning wurde bereits
ab den 70er Jahren entwickelt. Die generelle Idee
dahinter war -- ein Computer lehrt einen Nutzer
und spielt ihm relevante Informationen dann zu,
wenn sie von ihm benötigt werden.

Eines der ersten in der breiten Masse bekannten
Systeme dieser Art ist der in früheren Versionen
des Programms vorhandene \glqq Clippy \grqq .
Bei ihm handelte es sich um einen virtuellen
Lernassistenten für das Office-Paket.

Diese Idee dem Nutzer adaptives Lernen zur Verfügung zu
stellen scheiterte jedoch an der Umsetzung.

\section{Die Idee von adaptivem Lernen}
%% immer modernere Kenntnisse über Lernprozess (Behaviorismus -- Kognitivismus)
%% alte Lehrweisen (Frontalunterricht) eher Behavioristisch,
%% Wenn Lernverhalten individuell ist, muss auch Lehrverhalten
%% zugeschnitten auf das Individuum sein.


\section{Adaptivem Lernen im Vergleich mit dem klassischen Modell}
