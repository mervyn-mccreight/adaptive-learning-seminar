\chapter{Einführung}

Adaptives Lernen im e-Learning wurde bereits
ab den 70er Jahren entwickelt. Die generelle Idee
dahinter war -- ein Computer lehrt einen Nutzer
und spielt ihm relevante Informationen dann zu,
wenn sie von ihm benötigt werden.

Eines der ersten in der breiten Masse bekannten
Systeme dieser Art ist der in früheren Versionen
des Programms vorhandene \glqq Clippy \grqq .
Bei ihm handelte es sich um einen virtuellen
Lernassistenten für das Office-Paket.

Diese Idee dem Nutzer adaptives Lernen zur Verfügung zu
stellen scheiterte jedoch an der Umsetzung.

\section{Die Idee von adaptivem Lernen}
%% immer modernere Kenntnisse über Lernprozess (Behaviorismus -- Kognitivismus)
%% alte Lehrweisen (Frontalunterricht) eher Behavioristisch,
%% Wenn Lernverhalten individuell ist, muss auch Lehrverhalten
%% zugeschnitten auf das Individuum sein.


\section{Adaptivem Lernen im Vergleich mit dem klassischen Modell}
