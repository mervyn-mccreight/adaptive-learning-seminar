\chapter{Einführung}

\section{Die Idee von adaptivem Lernen}
Die Idee des adaptiven Lernen entstand mit der fortschreitenden
Erforschung des menschlichen Lernprozesses.
Der klassische Frontalunterricht, wie er von früher bekannt ist,
beruht stark auf dem behavioristischen Lernparadigma.
Das Lernen an sich ist ein statischer Prozess, der so angewandt
auf jeden Menschen die selbe Wirkung hat.

Eine modernere Ansicht wird durch den Kognitivismus
beschrieben. Hier rückt der Mensch als Individuum mehr in den Fokus.
Der generelle Ansatz ist hier, dass Lernen nicht mehr eine reine
Ansammlung von Wissenswiedergaben ist. Stattdessen lernt der Mensch
auf individuelle Art und Weise Problemstellungen anhand gewisser Regeln zu lösen.
Der Lernprozess ist hierbei individuell und dynamisch.

Diese Erkenntnisse über das Lernverhalten eines Menschen löste die Diskussion aus,
ob die bisher bekannten Lernmethoden auf die neuen Ansichten angepasst werden sollte.
Das gab den Anstoß zum Erforschen des adaptiven Lernens.
Adaptives Lernen soll diese Individualität berücksichtigen.
Anders als im reinen Frontalunterricht soll beim adaptiven Lernen
der Unterricht dynamisch auf den jeweiligen Lernenden angepasst werden.

\section{Adaptives Lernen im e-Learning}
Zu dieser Zeit existierten bereits Lernprogramme, die Nutzern
auf medialen Wege Wissen vermitteln sollten, ohne dass ein menschlicher Lehrer
zwingend notwendig ist. Diese basierten jedoch auf der veralteten behavioristischen
Ansicht und waren dementsprechend unflexibel konzipiert.

Ab den 70er Jahren wurde die Idee, adaptives Lernen auch auf Lernsysteme
zu übertragen, verfolgt. Die generelle Idee dahinter war -- ähnlich wie ein
menschlicher Lehrer soll der Computer einen Nutzer Wissen lehren und
ihm dabei relevante Informationen erst dann zugänglich machen, wenn sie von ihm
benötigt werden.

Eine bekannte, aber am der Umsetzung gescheiterte kleine Idee, adaptives Lernen
in normale Software zu integrieren, um dem Nutzer die Bedienung der Software zu lehren,
ist \glqq Clippy\grqq{} aus einer älteren Version des Office Pakets von Microsoft.

Im Folgenden sollen die Konzepte von adaptivem Lernen, sowie eine Art der
Umsetzung des Adaptiven Lernens im e-Learning erläutert werden.
Eine konkrete Art der Umsetzung in der Lernsoftware wird hierbei konzeptionell
diskutiert und anhand eines Beispiels gezeigt werden.
