\chapter{Zusammenfassung}
%% - anwendungsbereiche
%% -- obwohl schon lange existent (seit 80er jahren), nicht praxisrelevant
%% -- wenn verwendet, dann für
%% -- informatik (programmiersprachen lehren)
%% -- mathematik
%% -- medizin (diagnose von krankheiten)
%% - grund: wissen muss klar definierbar sein (wissensmodell)
%% - daher: anwendung in nicht klar definierbaren bereichen, wie philosophie schwierig
%%
%% - probleme auch:
%% -- komplexe entwicklung
%% -- dauer und kosten sehr hoch
%% -- wartungsaufwand hoch, wissen muss korrigiert/gepflegt werden
%% -- zu der zeit (80er jahre) nicht durchgesetzt, weil nur vor ort verwenbar
%% -- keine einfache verteilung über internet wie heute
%% -- wie oben erwähnt, wissensbereiche in denen its umsetzbar, beschränkt
%% -- forschung über das lernen selbst noch nicht abgeschlossen
%% --- daher: mangelndes verständnis des menschen
%% --- daher unmöglich, ein lernermodell zu definieren
%% --- konsequenz: lernermodell meist zu simpel
%% --- aber: adaptionsfähigkeit eines its ist abhängig von lernermodell
%% -- kritik auch: its übernehmen zu viel für den lernenden.
%% -- eigenaktivitätsanteil ist zu gering, daher ist das gelernte wissen flüchtiger
%% -- bisher auch kein sozialer kontext (ein lernender pro system zur zeit)
