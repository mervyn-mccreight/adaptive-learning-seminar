\chapter{Zusammenfassung}
%% - anwendungsbereiche
%% -- obwohl schon lange existent (seit 80er jahren), nicht praxisrelevant
%% -- wenn verwendet, dann für
%% -- informatik (programmiersprachen lehren)
%% -- mathematik
%% -- medizin (diagnose von krankheiten)
%% - grund: wissen muss klar definierbar sein (wissensmodell)
%% - daher: anwendung in nicht klar definierbaren bereichen, wie philosophie schwierig
Intelligente Adaptive Systeme sind ein interessanter Ansatz, um adaptives Lernen auf
Lernsoftware zu übertragen. Es wurden jedoch, trotz der langen Existenz dieser Medien,
wenig Fortschritte in diesem Bereich gemacht. Auch haben sich Intelligente Tutorielle Systeme
nicht gegenüber klassischer Lernsoftware durchgesetzt.

\begin{table}[htbp]
	\centering
	\begin{tabular}{c}
		\textbf{Bekannte Intelligente Tutorielle Systeme}
	\end{tabular}

	\begin{tabular}{m{3cm} || c | c | c}
		\hline
		\textbf{Name} & \textbf{Jahr der Entwicklung} & \textbf{Fachbereich} & \textbf{Zusatzinformationen} \\
		\hline
		SCHOLAR & 1970 & Geographie \\
		GEO Tutor & 1989 & Geographie \\
		MYCIN & 1976 & Medizin & Identifikation von Infektionen \\
		LISP-Tutor & 1985 & Informatik & LISP-Programmierung \\
		BRIDGE & 1988 & Informatik & Pascal-Programmierung \\
		PROUST & 1988 & Informatik & Pascal-Programmierung \\
		Algebraland & 1985 & Mathematik \\
		Quadratic Tutor & 1982 & Mathematik \\
		WEST & 1979 & Mathematik \\
	\end{tabular}

	\caption[Auszug aus einer Übersicht über Intelligenter Tutorieller Systeme]{Auszug aus einer Übersicht über Intelligenter Tutorieller Systeme \\ Quelle: \cite[S. 192f]{schulmeister2002}}
\end{table}

In der Tabelle kann man sehen, dass die meisten ITS in den 80er Jahren entwickelt wurden.
Betrachtet man die jeweiligen Fachegebiete, fällt auf, dass die Gesamtmenge der Fachbereiche
sehr eingeschränkt ist. Die meisten ITS behandeln Themen wie Informatik, Mathematik, Physik
oder Medizin. Der Grund hierfür ist, dass für eine geeignete Wissensrepräsentation in der Form,
wie sie im Wissensmodell vorliegen muss, das Wissen in der jeweiligen Thematik sehr eindeutig
definiert sein muss. So lassen sich Mehrdeutigkeiten, oder besonders subjektiv interpretierbare
Sachverhalte schlecht maschinell automatisiert bewerten oder logisch erfassen. Daher ist eine Programmierung
eines Intelligenten Tutoriellen Systems in solchen schwach definierten Thematiken, wie zum Beispiel
der Philosophie, schwer umsetzbar.

%% -- komplexe entwicklung
%% -- dauer und kosten sehr hoch
%% -- wartungsaufwand hoch, wissen muss korrigiert/gepflegt werden
%% -- zu der zeit (80er jahre) nicht durchgesetzt, weil nur vor ort verwenbar
%% -- keine einfache verteilung über internet wie heute
Ein weiterer Punkt, der die Verbreitung von ITS negativ beeinflusst ist die vergleichsweise
aufwendige und komplexe Entwicklung und Wartung. Dadurch sind sowohl die Dauer, und die
damit verbundenen Kosten zur Umsetzung eines ITS sehr hoch. Gerade zur Zeit der 80er Jahre,
in denen sich die meisten ITS entwickelt haben, gab es noch nicht die jetzigen Möglichkeiten zur
Verbreitung entwickelter Software. So war ein entwickeltes ITS später nur lokal an der Entwicklungsstelle
verwendbar.

%% -- forschung über das lernen selbst noch nicht abgeschlossen
%% --- daher: mangelndes verständnis des menschen
%% --- daher unmöglich, ein lernermodell zu definieren
%% --- konsequenz: lernermodell meist zu simpel
%% --- aber: adaptionsfähigkeit eines its ist abhängig von lernermodell
Die Forschung, die sich damit beschäftigt den Lernprozess oder auch den Wahrnehmungsprozess von
Menschen zu verstehen, gilt auch heute nicht als abgeschlossen. Das bedeutet, dass auch zum jetzigen Stand
das Verständnis über das Lernverhalten eines Menschen nicht vollständig ist.
In Folge dessen ist es schwierig, ein geeignetes Lerner- und auch Tutorenmodell zu schreiben,
dass entsprechende Maßnahmen bezogen auf einen Menschen treffen soll. Ein ITS steht und fällt mit der
korrekten Adaption. Diese ist jedoch abhängig von der Qualität des Lerner- und Tutorenmodells.
Die Konsequenz von der unvollständigen Forschung ist, dass das Lernermodell meist zu simpel konstruiert wird.
Auf diese Weise wird die Adaptionsfähigkeit eines ITS negativ beeinflusst und ein eigentlich gutes Konzept
kann sein volles Potential nicht enfalten.

%% - probleme auch:
%% -- kritik auch: its übernehmen zu viel für den lernenden.
%% -- eigenaktivitätsanteil ist zu gering, daher ist das gelernte wissen flüchtiger
%% -- bisher auch kein sozialer kontext (ein lernender pro system zur zeit)
Auch der generelle Ansatz des adaptiven Lernens ist nicht frei von Kritik. Oftmals wird bemängelt, dass durch eine
zu hohe Anpassung an die individuellen Bedürfnisse eines Lernenden der Eigenanteil des Schülers
im Lernprozess viel zu gering ausfällt. Dabei soll gerade die Eigeninitiative
dafür sorgen, dass gelerntes Wissen sich besser festigt und nicht mehr so flüchtig ist.

Die meisten Intelligenten Tutoriellen Lernsysteme berücksichtigen auch keine soziale Komponente, die in der Lerntheorie
oftmals jedoch eine große Rolle spielt.

Diese hohe Anzahl an Kritikpunkten, und sowohl die mit der Entwicklung verbundenen hohen Kosten als auch der Mehraufwand,
sind mögliche Begründungen für das bisherige Scheitern der Intelligenten Tutoriellen Systeme im Vergleich
zu herkömmlicher Lernsoftware. Gerade in der Wirtschaft wiegen die Vorteile von ITS die Mehrkosten bisher nicht auf.
