\chapter{Intelligente Tutorielle Systeme}
Intelligente Tutorielle Systeme\footnote{kurz ITS} sind eine um 1973 von Derek. H. Sleeman
und J.R. Hartley definierte Art von computergesteuerten Lernprogrammen.
Diese Art von Lernprogrammen waren der erste Ansatz, das in der Lerntheorie
ergründete Adaptive Lernen im softwaregestützten e-Learning zu etablieren,
um die Effizienz von Lernsoftware zu verbessern.

Intelligente Tutorielle Systeme erreichen ihre Adaptivität und Flexibilität durch
eine individuell an den Benutzer angepasste Art von Lernangeboten.
Das Verhalten während des Lernens, sowie die Leistungen während Lernüberprüfungen,
oder interaktiven Aufgaben, werden bewertet, um die Präsentation der Lerninhalte zu wählen.
Das bedeutet, dass ein Intelligentes Tutorielles System zu jeder Zeit versucht zu erkennen,
wie ausgeprägt das Wissen eines Anwenders in der jeweiligen Thematik ist, um die
zu vermittelnden Inhalte dementsprechend anzupassen, die dazu führen sollen ein
definiertes Lernziel zu erreichen.
So wird ein Anwender, der bisher keine Erfahrungen in einem Thema hat, nicht sofort
mit komplexen Sachverhalten konfrontiert, sondern langsam in die Thematik eingeführt,
bis er für die höheren Lernmaterialien bereit ist.


\section{Definition}
\glqq Intelligente tutorielle Systeme (ITS) sind adaptive Mediensysteme, die sich ähnlich
einem menschlichen Tutor an die kognitiven Prozesse des Lernenden anpassen
sollen, indem sie die Lernfortschritte und -defizite analysieren und dementsprechend
das Lernangebot generativ modifizieren sollen.\grqq{} \cite[S. 555]{issing2002information}

Die Intelligenz eines Intelligenten Tutoriellen Systems besteht dementsprechend
in der Adaption der Lehrinhalte an den Wissensstand des jeweiligen Benutzers.
Ein ITS versucht, vergleichbar mit einem menschlichen Lehrer, einen flexiblen und adaptiven
Dialog mit dem Lernenden zu führen, indem es den Unterricht den Merkmalen und Fortschritten
des Benutzers anpasst.

Signifikant für ein Intelligentes Tutorielles System sind hierbei folgende drei Hauptmerkmale:

\begin{description}
	\item[Adaptivität]
  Adaptivität beschreibt die Fähigkeit des Systems, sich selbstständig an den
  jeweiligen Benutzer anzupassen. Dies geschieht durch die Auswertung von
  Informationen über zur Verfügung stehenden Lerninhalten, Bewertung des Lernenden, sowie
  der Anwendung von definierten pädagogischen Strategien.
  Vergleichbar ist dies mit einer typischen Situation, mit der sich ein
  menschlicher Lehrer bei der Gestaltung seines Unterrichts konfrontiert sieht.
  Einem Lehrer ist es nicht möglich, während der Vorbereitung seines Unterrichts
  zu wissen, welche Strategien er später im Unterricht benötigen wird, um
  das zu übermittelnde Wissen optimal zu erklären. Er ist dazu gezwungen sich
  im Laufe des Unterrichts dynamisch an die Situation anzupassen.

	\item[Flexibilität]
  Die Flexibilität des Systems bezieht sich auf die Fähigkeit, die Darstellung
  der Lerninhalte zu verändern. Diese Fähigkeit wird durch die getrennte
  Realisierung der Wissensbasis und der tutoriellen Komponente ermöglicht.
  Diese beiden Begriffe werden im Laufe dieser Ausarbeitung näher erläutert.

	\item[Diagnosefähigkeit]
  Die Diagnosefähigkeit ist ein weiterer Kernaspekt eines Intelligenten Tutoriellen Systems.
  Sie beschreibt die Fähigkeit, den aktuellen Wissensstand, sowie weitere Kriterien
  des Lernenden zu analysieren, um so Rückschlüsse über seine themen- und lernspezifische
  Kompetenz zu bewerten.
  Auf diese Art und Weise versucht ein Intelligentes Tutorielles System ein Modell des
  Lernenden abzuleiten, um darauf basierend eine passende individuelle Lehrstrategie
  für den Lernenden zu entwickeln.
  Ohne diese Fähigkeit wäre ein ITS nicht dazu in der Lage,
  seine Inhalte auf eine sinnvolle Art und Weise individuell an einen Lerner anzupassen.
\end{description}

Wichtig ist, dass der Lernablauf weiterhin benutzergesteuert ist. Der Benutzer steuert
selbst, in welcher Geschwindigkeit er seinen Lernprozess gestaltet.
Das Intelligente Tutorielle System bietet dem Benutzer hierbei jedoch nur zu
der Bewertung seines Wissensstand passende Lernmaterialen an, um mit dem Lernen
fortzufahren. So soll gewährleistet werden, dass er Lernende das Lernziel auf einem
für ihn optimalen Weg erreicht.

\section{Unterschiede zu klassischen tutoriellen Systemen}
Um zu verstehen, wodurch sich Intelligente Tutorielle Systeme von den früheren
klassischen Tutoriellen Systemen unterscheiden, muss zunächst die Funktionsweise
von klassischen Tutoriellen Systemen erörtert werden.

Bei klassischen Tutoriellen Systemen handelt es sich ebenfalls um Lernsoftware,
die einem Lernenden auf (multi)mediale Art und Weise Lehrstoff präsentiert, um ein
definiertes Lernziel zu erreichen. Es handelt sich hierbei jedoch nicht um reine
Präsentationssysteme.

\begin{figure}
	\centering
	\begin{tikzpicture}[
				part/.style={rectangle, minimum width=4cm, minimum height=2cm, very thick, draw=black, font=\itshape}
				]
		\node (start) [part] {Programmstart};
		\node (presentation) [part, right=of start] {Lehrstoffpräsentation};
		\node (question) [part, right=of presentation] {Fragestellung};
    \node (end) [part, below=of start] {Programmende};
    \node (feedback) [part, below=of presentation, right=of end] {Feedback};
		\node (analyze) [part, below=of question, right=of feedback] {Analyse der Antwort};


		\draw [->] (start) -- (presentation);
		\draw [->] (presentation) -- (question);
		\draw [->] (question) -- (analyze);
		\draw [->] (analyze) -- (feedback);
		\draw [->] (feedback) -- (presentation);
		\draw [->] (feedback) -- (end);
	\end{tikzpicture}

	\caption{Prinzip eines klassischen tutoriellen Systems}
\end{figure}

In der Abbildung ist erkennbar, dass tutorielle Systeme zusätzlich zur
reinen Präsentation der Lehrinhalte, zwischendurch einige Fragen an den Lernenden stellen.
Die Antworten auf diese Fragen, die der Lernüberprüfung dienen, beeinflussen den weiteren
Verlauf des Lernkurses. Wichtig hierbei ist, dass lediglich der Ablauf des Lernkurses
beeinflusst wird. So wird ein Lehrinhalt bei unzureichendem Ergebnis in der Leistungsüberprüfung
solange wiederholt, bis der Lernende dazu in der Lage ist, die Fragestellung korrekt zu beantworten.
In einem tutoriellen System erhält der Lernende üblicherweise sofort Feedback auf
seine erbrachte Leistung.
Bei einer falschen Antwort kann das beispielsweise die Angabe der korrekten Lösung,
oder ein Hinweis auf den richtigen Lösungsweg sein. Dieses Feedback simuliert
hierbei die Rolle eines Tutors.

Klassische tutorielle Systeme lassen sich in der Regel dem behavioristischem Lernparadigma
zuordnen. Die Software repräsentiert eine absolute Lehrautorität, die dem Lernenden
Wissen präsentiert. Die Bewertung von Antworten beschränkt sich auf falsch oder richtig.

Ein einfaches Beispiel für ein klassisches tutorielles System ist das Lernen für
die theoretische Fahrschulprüfung mit Hilfe einer Handy-Applikation. Klassisch werden
hierbei die aus den offiziellen Fragebögen bekannten Fragen repetetiv präsentiert,
und der Benutzer dazu aufgefordert, eine Antwort auszuwählen.

\begin{figure}
	\centering
    \includegraphics[width=0.7\textwidth]{bilder/fahrschule_app.jpg} %70% der Textbreite
	\caption{Beispielbild der Pocket Fahrschule Handy-Applikation}
\end{figure}

Tut er dies, wird ihm unmittelbar nach der Eingabe der Antwort vermittelt,
ob diese richtig oder falsch war.
Bei komplexeren Fragestellungen wird, je nach Applikation, zusätzlich erläutert warum
die korrekte Lösung korrekt ist.
Nach diesem Verfahren wird fortgefahren, bis der Lernende alle vorhandenen Fragen
korrekt beantwortet hat.

Intelligente Tutorielle Systeme versuchen darüber hinaus die Erkenntnisse neuerer Lernparadigmen,
wie zum Beispiel dem Kognitivismus zu berücksichtigen.
Im Gegensatz zu einem klassischen System kann ein ITS individuelle Kritik an einen Lernenden
formulieren. Der präsentierte Lehrinhalt ist nicht statisch, also in jedem Fall gleich, sondern
angepasst an die individuellen Bedürfnisse eines Lernenden.
Auf diese Weise sind ITS dazu in der Lage auch komplexere Sachverhalte zu vermitteln.

\section{Struktur}
Wie jede Software entsprechen auch Intelligente Tutorielle Systeme einer klar definierten Architektur.
In der Abbildung ist zu sehen, dass ein Intelligentes Tutorielles System aus vier
Hauptkomponenten besteht, die getrennt voneinander implementiert werden.

\begin{figure}[!ht]
	\centering
    \includegraphics[width=0.7\textwidth]{bilder/its_structure.jpg} %70% der Textbreite
	\caption{Struktur eines Intelligenten Tutoriellen Systems}
\end{figure}

%% TODO: das Schaubild mit tikz nachstellen wäre schöner.
%%\begin{figure}
%%\centering
%%
%%	\begin{tikzpicture}[node distance=2.2cm]
%%
%%		\tikzstyle{comp1} = [draw, ellipse, rounded corners, minimum height=1cm, minimum width=4cm, fill=gray!50]
%%		\tikzstyle{comp2} = [draw, rectangle, rounded corners, minimum height=1cm, minimum width=4cm]
%%		\tikzstyle{comp3} = [draw, ellipse, minimum height=1cm, minimum width=4cm, fill=gray!10, text centered]
%%
%%		\tikzstyle{arrow} = [thick,->,>=stealth]
%%
%%		%%% NODES %%%
%%		\node (tutand)     [comp1]                 {Lernender};
%%		\node (ui)     		 [comp2, below of=tutand, minimum width=10cm]                 {Benutzerschnittstelle};
%%		\coordinate[below of=ui] (c);
%%
%%		\node (tutor)         [comp3, left of=c]      {Tutorenmodell};
%%		\node (learn)        [comp3, right of=c]     {Lernermodell};
%%		\node (knowledge)       [comp2, below of=c,minimum width=10cm]     {Wissensmodell};
%%
%%		%%% ARROWS %%%
%%		\draw [arrow] (tutand) -- (ui);
%%
%%		\draw [arrow] (ui) -- (tutand);
%%		\draw [arrow] ([yshift=-0.5cm, xshift=-2.8cm]ui.east) -- (learn);
%%
%%		\draw [arrow] (learn) -- (tutor) node [midway, fill=white, text width=3cm] {wahrgenommener Wissensstand,\linebreak Verhaltenssequenzen des Lernenden};
%%
%%		\draw [arrow] (tutor) -- (knowledge);
%%
%%		\draw [arrow] (knowledge) -- (tutor);
%%		\draw [arrow] (knowledge) -- (learn);
%%	\end{tikzpicture}
%%	\caption{Struktur eines Intelligenten Tutoriellen Systems}
%%\end{figure}

Die Kanten des Graphen, unter hinzunahme ihrer Beschriftungen, beschreiben die Art der
Kommunikation zwischen den einzelnen Komponenten.
Man sieht, dass die einzige Kommunikation zwischen dem Lernenden und dem System über die
bereitgestellte Benutzerschnittstelle statt findet. Ein beispielhafter Ablauf könnte sein:

Der Schüler löst eine Aufgabe. Über die Benutzerschnittstelle wird dessen Lösung
an das Lernermodell weitergeleitet. Das Lernermodell erhält gleichermaßen vom Wissensmodell
die Musterlösung der zu lösenden Aufgabe. An dieser Stelle werden die beiden Lösungsansätze
verglichen, um mit Hilfe dieses Vergleichs den Wissensstand und die Verhaltensweisen
des Lernenden abzuleiten. Das Ergebnis dieser Analyse wird dem Tutorenmodell mitgeteilt, welches
auf Basis der Analyse geeignete pädagogische Lernstrategien für den Lernenden einschlagen kann,
und ihm entsprechend seiner Bedürfnisse Feedback liefern kann. Dieses wird dem Lernenden
über die Benutzerschnittstelle kommuniziert. Im weiteren Verlauf der Ausarbeitung
wird nun genauer auf die einzelnen Komponenten eingegangen.

\subsection{Das Wissensmodell}
- auch Expertenmodell
- bildet Wissensbasis des Lernsystems.
- Ansammlung von Kenntnissen, Erfahrungen, Methoden und Allgemeinwissen.
- Verkörpert drei verschiedene Wissenskategorien:
-- Deklaratives Wissen (Faktenwissen, "Was-Wissen")
--- "Vokabelwissen" -> Definition von notwendigen Begriffen, etc. (z.B. im
Kontext Mathe - was ist eine Wurzel, was ist Addieren...)
-- Prozedulares Wissen (praktisches Wissen, "Wie-Wissen")
--- Beinhaltet Regeln, mit deren Hilfe sich Problemstellungen lösen lassen
--- Regelwissen (Schema, nach dem ich vorgehen muss zum Addieren ...)
-- Heuristisches Wissen (Erfahrungswissen)
--- Beinhaltet Erfahrungswerte von Experten, bspw. um Bekannte Fehler zu erkennen und daraufhin
--- bekannte Tipps geben zu können.
--- Handlungsempfehlungen
--- Dient der Unterstützung zur Findung einer richtigen Herangehensweise an ein Problem

- Kann auf zwei versch. Arten repräsentiert werden.
-- Black Box Modell
--- Vorgehensweisen vom Programm sind verborgen
--- Nur Ergebnisse von Lösungen sind einsehbar
--- Bildet keine menschliche Intelligenz nach, daher wird Lösungsweg verschleiert.
--- Kann kompliziertere Fragestellungen schneller berechnen, da kein Wert auf Nachvollziehbarkeit gelegt wird.
-- Glass Box Modell
--- Jeder Einzelschritt ist einsehbar.
--- Erhebt Anspruch menschen vergleichbare Methoden zur Problemlösung anzuwenden
--- Vollständiger Lösungsweg ist jederzeit sichtbar, Lernender kann interaktiv Fragen Stellen
--- Modell kann Lernenden zu jeder Zeit Schritt für Schritt helfen.
- Formell harte Abgrenzung, oft werden die Formen vermischt.
- Wissen meist in Form von semantischen Netzen modelliert
-- Wie ein Graph/Baum
-- Knoten repräsentieren Wissenseinheiten
-- Kanten repräsentieren Verknüpfungen zwischen Wissen
-- Üblicherweise in einer Hierarchie
--- Wissen von einer Ebene ist Voraussetzung, für verbundenes Wissen in nächster Ebene.
\subsection{Das Lernermodell}

\subsection{Das Tutorenmodell}

\subsection{Die Benutzerschnittstelle}

\section{Formen der Modifikation zur Adaption}

\subsection{Sequenzierung}

\subsection{Analyse von Ergebnissen}

\subsection{Unterstützung beim Lösen von Problemen}

\subsection{Adaptive Präsentation}

\subsection{Adaptive Navigation}
