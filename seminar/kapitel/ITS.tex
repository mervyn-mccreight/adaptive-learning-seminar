\chapter{Intelligente Tutorielle Systeme}
Intelligente Tutorielle Systeme\footnote{kurz ITS} sind eine um 1973 von Derek. H. Sleeman
und J.R. Hartley definierte Art von computergesteuerten Lernprogrammen.
Diese Art von Lernprogrammen waren der erste Ansatz, das in der Lerntheorie
ergründete Adaptive Lernen im softwaregestützten e-Learning zu etablieren,
um die Effizienz von Lernsoftware zu verbessern.

Intelligente Tutorielle Systeme erreichen ihre Adaptivität und Flexibilität durch
eine individuell an den Benutzer angepasste Art von Lernangeboten.
Das Verhalten während des Lernens, sowie die Leistungen während Lernüberprüfungen,
oder interaktiven Aufgaben, werden bewertet, um die Präsentation der Lerninhalte zu wählen.
Das bedeutet, dass ein Intelligentes Tutorielles System zu jeder Zeit versucht zu erkennen,
wie ausgeprägt das Wissen eines Anwenders in der jeweiligen Thematik ist, um die
zu vermittelnden Inhalte dementsprechend anzupassen, die dazu führen sollen ein
definiertes Lernziel zu erreichen.
So wird ein Anwender, der bisher keine Erfahrungen in einem Thema hat, nicht sofort
mit komplexen Sachverhalten konfrontiert, sondern langsam in die Thematik eingeführt,
bis er für die höheren Lernmaterialien bereit ist.


\section{Definition}
\glqq Intelligente tutorielle Systeme (ITS) sind adaptive Mediensysteme, die sich ähnlich
einem menschlichen Tutor an die kognitiven Prozesse des Lernenden anpassen
sollen, indem sie die Lernfortschritte und -defizite analysieren und dementsprechend
das Lernangebot generativ modifizieren sollen.\grqq{} \cite[S. 555]{issing2002information}

Die Intelligenz eines Intelligenten Tutoriellen Systems besteht dementsprechend
in der Adaption der Lehrinhalte an den Wissensstand des jeweiligen Benutzers.
Ein ITS versucht, vergleichbar mit einem menschlichen Lehrer, einen flexiblen und adaptiven
Dialog mit dem Lernenden zu führen, indem es den Unterricht den Merkmalen und Fortschritten
des Benutzers anpasst.

Signifikant für ein Intelligentes Tutorielles System sind hierbei folgende drei Hauptmerkmale:

\begin{description}
	\item[Adaptivität]
  Adaptivität beschreibt die Fähigkeit des Systems, sich selbstständig an den
  jeweiligen Benutzer anzupassen. Dies geschieht durch die Auswertung von
  Informationen über zur Verfügung stehenden Lerninhalten, Bewertung des Lernenden, sowie
  der Anwendung von definierten pädagogischen Strategien.
  Vergleichbar ist dies mit einer typischen Situation, mit der sich ein
  menschlicher Lehrer bei der Gestaltung seines Unterrichts konfrontiert sieht.
  Einem Lehrer ist es nicht möglich, während der Vorbereitung seines Unterrichts
  zu wissen, welche Strategien er später im Unterricht benötigen wird, um
  das zu übermittelnde Wissen optimal zu erklären. Er ist dazu gezwungen sich
  im Laufe des Unterrichts dynamisch an die Situation anzupassen.

	\item[Flexibilität]
  Die Flexibilität des Systems bezieht sich auf die Fähigkeit, die Darstellung
  der Lerninhalte zu verändern. Diese Fähigkeit wird durch die getrennte
  Realisierung der Wissensbasis und der tutoriellen Komponente ermöglicht.
  Diese beiden Begriffe werden im Laufe dieser Ausarbeitung näher erläutert.

	\item[Diagnosefähigkeit]
  Die Diagnosefähigkeit ist ein weiterer Kernaspekt eines Intelligenten Tutoriellen Systems.
  Sie beschreibt die Fähigkeit, den aktuellen Wissensstand, sowie weitere Kriterien
  des Lernenden zu analysieren, um so Rückschlüsse über seine themen- und lernspezifische
  Kompetenz zu bewerten. Ohne diese Fähigkeit wäre ein ITS nicht dazu in der Lage,
  seine Inhalte auf eine sinnvolle Art und Weise individuell an einen Lerner anzupassen.
\end{description}

Wichtig ist, dass der Lernablauf weiterhin benutzergesteuert ist. Der Benutzer steuert
selbst, in welcher Geschwindigkeit er seinen Lernprozess gestaltet.
Das Intelligente Tutorielle System bietet dem Benutzer hierbei jedoch nur zu
der Bewertung seines Wissensstand passende Lernmaterialen an, um mit dem Lernen
fortzufahren. So soll gewährleistet werden, dass er Lernende das Lernziel auf einem
für ihn optimalen Weg erreicht.

\section{Unterschiede zu klassischen tutoriellen Systemen}

\begin{figure}
	\centering
	\begin{tikzpicture}[
				part/.style={rectangle, minimum width=4cm, minimum height=2cm, very thick, draw=black, font=\itshape}
				]
		\node (start) [part] {Programmstart};
		\node (presentation) [part, right=of start] {Lehrstoffpräsentation};
		\node (question) [part, right=of presentation] {Fragestellung};
    \node (end) [part, below=of start] {Programmende};
    \node (feedback) [part, below=of presentation, right=of end] {Feedback};
		\node (analyze) [part, below=of question, right=of feedback] {Analyse der Antwort};


		\draw [->] (start) -- (presentation);
		\draw [->] (presentation) -- (question);
		\draw [->] (question) -- (analyze);
		\draw [->] (analyze) -- (feedback);
		\draw [->] (feedback) -- (presentation);
		\draw [->] (feedback) -- (end);
	\end{tikzpicture}

	\caption{Ablaufdiagramm eines klassischen tutoriellen Systems}
\end{figure}

\section{Struktur}

\subsection{Das Wissensmodell}

\subsection{Das Lernermodell}

\subsection{Das Tutorenmodell}

\subsection{Die Benutzerschnittstelle}

\section{Formen der Modifikation zur Adaption}

\subsection{Sequenzierung}

\subsection{Analyse von Ergebnissen}

\subsection{Unterstützung beim Lösen von Problemen}

\subsection{Adaptive Präsentation}

\subsection{Adaptive Navigation}
