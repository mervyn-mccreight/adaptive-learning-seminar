\chapter{Adaptives Lernen in der Lerntheorie}
TODO: Irgendein Einleitungstext.

\section{Aptitude Treatment Interaction}
Das Konzept der Adaptivität basiert prinzipiell auf der Forschung des Aptitude-Treament-Interaktion Paradigmas.
Hierbei handelt es sich um einen Ansatz zur Instruktion von Lernenden.
Er besagt, dass eine Anpassung Lehrmethode an das Niveau der individuellen Lernfähigkeiten
des Lernenden notwendig sind, um einen best möglichsten Lerneffekt zu erzielen.
Das bedeutet, dass sich ausgehend von den Ausgangsvoraussetzungen \footnote{engl. Aptitude}
in unterschiedlichen Lernumgebungen \footnote{engl. Treatments} unterschiedliche Lernerfolge zeigen.
Forschungen bezüglich der Aptitude-Treatment-Interaktion zielen darauf ab, Informationen
zur liefern, mit deren Hilfe es möglich ist, einzuordnen, welche Unterrichtsform sich
für welche individuellen Voraussetzungen und Merkmale am besten eignen.\cite[S. 203]{krohne2007psychologische}

So stellte sich in den Forschungen heraus, dass Lerner mit niedrigerem Kenntnisstand und erhöhtem Angstniveau bezogen auf
eine Unterrichtssituation in höherem Maße von einer hochstrukturierten Unterrichtsform mit vielen festen Vorgaben profitieren,
als leistungsstärkere Lerner. Diese profitierten eher von einem gegensätzlichen Unterrichtsmodell mit vielen Freiheitsgraden.
\cite[S.65]{hasselhornlernverlaufsdiagnostik}

\section{Adaptivität von Lernen}
Auf Basis des Aptitude-Treatment-Interaktion Ansatzes entwickelte sich das Konzept des Adaptiven Lernens.
Hierbei handelt es sich um eine Lehrform, deren Ziel es ist die Unterrichtsform möglichst optimal an die Lernvoraussetzungen
des Lerners anzupassen, um die Effektivität des Lernens zu steigern. Konträr zum klassischen Frontalunterricht handelt
es sich hierbei also um eine Lehrform, die sich sehr auf den Lerner als Individuum konzentriert.

\section{Adaptionsmaßnahmen}
Um eine Lernsituation adaptiv zu gestalten, werden Maßnahmen angewandt, die sich grundsätzlich in zwei
unterschiedliche Kategorien einteilen lassen. So unterscheidet man zwischen Maßnahmen auf der Makro- und Mikroebene.

Maßnahmen in der Makroebene einer Lernumgebung beschreiben Aktionen auf Klassenebene.
So werden zum Beispiel Gruppen nach Leistungsniveau eingeteilt, der Lernplan im Gesamten für
diese Einteilungen gruppenindividuell angepasst. Eine weitere Maßnahme auf der Makroebene beschreibt
die Einführung von kooperativem Lernen. So wird invididuelles Wissen der Lernenden über einen sozialen Austausch
über eine bestimmte Thematik revidiert, integriert, neu organisiert oder weiter ausdifferenziert.

Dagegen stehen Maßnahemn, die eine direkte Interaktion und/oder Kommunikation zwischen Lehrer und Lernendem beschreiben.
Diese werden als Maßnahmen in der Mikroebene bezeichnet. Durch die Beschäftigung mit einem Lernenden
ist es dem Lehrer möglich, besser auf die individuellen Stärken und Schwächen eines Lernenden einzugehen.
Das Ziel hierbei ist, den Unterrichtsinhalt und die Lernmaßnahmen besser auf jeden einzelen Lernenden anzupassen.

\section{Adaptionszwecke}
Betrachtet man die Frage Wie und Warum Adaptives Lernen eingesetzt wird, lassen sich drei Adaptionszwecke erkennen.
Diese stellen dar, auf welche Aspekte des Lernenden als Individuum geachtet und eingegangen wird, um die
Qualität und Effektivität des Lernens zu verbessern.
Es wird zwischen dem Fördermodell, dem Kompensationsmodell und dem Präferenzmodell unterschieden.

\subsection{Fördermodell}
Das Ziel des Fördemodells ist es, Lerndefizite des Lernenden zu beseitigen.
Meist wird dies durch eine Anpassung in Form von zusätzlichen Lerneinheiten erreicht.
Hierfür müssen die Lerndefizite beispielsweise über zusätzliche Tests oder Prüfungen erkannt
und beseitigt werden. \cite[S. 19]{lehmann2010lernstile}

\subsection{Kompensationsmodell}
Das Kompensationsmodell richtet sich auf Lernende mit unzureichenden Lernvoraussetzungen.
Diese können beispielsweise durch Überforderung oder eine generell niedrige Motivation entstehen.
Hier wird versucht, dem Lernenden durch geeignete Hilfestellung abhilfe zu leisten, um die Defizite zu kompensieren.
\cite[S. 19]{lehmann2010lernstile}

\subsection{Präferenzmodell}
Anders als in den beiden vorherigen Modellen, in denen auf die Schwächen eines Lernenden eingegangen wird,
sollen im Präferenzmodell die Stärken eines Lernenden nutzbar gemacht werden. Ist erkennbar, dass
der Lernende in einem bestimmten Bereich über besondere Lernvoraussetzungen verfügt, sollen diese genutzt werden,
um den Lernprozess für diesen Lernenden zu optimieren.
\cite[S. 19]{lehmann2010lernstile}

\section{Definitionsebenen von Adaptivität}
