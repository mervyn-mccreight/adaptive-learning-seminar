\chapter{Adaptives Lernen in der Lerntheorie}
Je länger das Verständnis von der Fähigkeit eines Menschen zu lernen Gegenstand der Forschung ist,
desto mehr gerät der Mensch als Individuum auch in den verschiedenen Lehrtheorien in den Vordergrund.
So wird in einer der ältesten Lerntheorien des Menschen, dem Behaviorismus, das Gehirn des Menschen noch als
Black-Box gesehen, wodurch die Individualität des Lernens eines Menschen fast vollständig außer Acht gelassen wird.
Erst in späteren Lerntheorien wie dem Konstruktivismus stellte sich heraus, dass der Lernprozess
theoretisch für jeden Menschen individuell unterschiedlich sein kann.
Aufgrund dieser Erkenntnisse erscheint es logisch, dass auch die Form des Lehrens auf den Menschen als Individuum eingehen sollte.

\begin{table}[!htbp]
	\centering
	\begin{tabular}{c}
		\textbf{Vergleich Behaviorismus und Kognitivismus}
	\end{tabular}

	\begin{tabular}{m{3cm} || c | c}
		\hline
		\  & \textbf{Behaviorismus} & \textbf{Kognitivismus} \\
		\hline
		\textbf{Hirn is} & passiver Behälter & Informationsverarbeitend \\
		\textbf{Wissen ist} & Input-Output Relation & interner Verarbeitungsprozess \\
		\textbf{Paradigma} & Stimulus-Response & Problemlösung  \\
		\textbf{Strategie} & Lehren & Beobachten und Helfen \\
		\textbf{Lehrer ist} & Autorität & Tutor \\
		\textbf{Interaktion} & starr & dynamisch, abhängig von Tutorand \\
	\end{tabular}

	\caption[Vergleich Behaviorismus und Kognitivismus]{Vergleich Behaviorismus und Kognitivismus \\ Quelle: \cite[S. 110, S. 174]{baumgartner1994}}
  \label{tab:behaviorismus_kognitivismus}
\end{table}

Was heißt hierbei adaptiv\footnote{lat. adaptare}?
Das Wort adaptiv beschreibt die Fähigkeit der Anpassung.
Es geht bei adaptivem Lernen also um angepasste Wissensvermittlungsformen.

Adaptives Lernen in der Lerntheorie beschreibt die Anpassung von Lehrumgebungen
an die Bedürfnisse eines Lernenden oder einer Gruppe von Lernenden, um die Lernsituation
zu verbessern. Ziel von Adaptivem Lernen ist es, den Unterricht in der Art anzupassen,
dass ein Lernender genau die Information und das Wissen vermittelt bekommt, dass für ihn relevant ist,
um das Thema zu verstehen. Im optimalen Fall wird das Wissen jedem Lernenden
in der für ihn individuell am besten geeignesten Form präsentiert.

Vereinfacht ausgedrückt ist adaptives Lernen nichts anderes, als das, was ein guter Lehrer
spontan immer anwenden sollte: Auf seine Schüler im Einzelnen einzugehen.

\section{Aptitude Treatment Interaction}
Das Konzept der Adaptivität basiert auf der Forschung des Aptitude-Treament-Interaktion Paradigmas.
Hierbei handelt es sich um einen Ansatz zur Instruktion von Lernenden.
Er besagt, dass eine Anpassung der Lehrmethode an das Niveau der individuellen Lernfähigkeiten
des Lernenden notwendig ist um den bestmöglichsten Lerneffekt zu erzielen.
Das bedeutet, dass sich ausgehend von den Ausgangsvoraussetzungen \footnote{engl. Aptitude}
in unterschiedlichen Lernumgebungen \footnote{engl. Treatments} unterschiedliche Lernerfolge zeigen.
Forschungen bezüglich der Aptitude-Treatment-Interaktion zielen darauf ab, Informationen
zu liefern, mit deren Hilfe es möglich ist, einzuordnen, welche Unterrichtsformen sich
für welche individuellen Voraussetzungen und Merkmale am besten eignen.\cite[S. 203]{krohne2007psychologische}

So stellte sich in Forschungen heraus, dass Lerner mit niedrigerem Kenntnisstand und erhöhtem Angstniveau bezogen auf
eine Unterrichtssituation in höherem Maße von einer hochstrukturierten Unterrichtsform mit vielen festen Vorgaben profitieren,
als leistungsstärkere Lerner. Diese profitierten eher von einem gegensätzlichen Unterrichtsmodell mit vielen Freiheitsgraden.
\cite[S.65]{hasselhornlernverlaufsdiagnostik}

\section{Adaptivität von Lernen}
Auf Basis des Aptitude-Treatment-Interaktion Ansatzes entwickelte sich das Konzept des adaptiven Lernens.
Hierbei handelt es sich um eine Lehrform, deren Ziel es ist, die Unterrichtsform möglichst optimal an die Lernvoraussetzungen
des Lerners anzupassen, um die Effektivität des Lernens zu steigern. Konträr zum klassischen Frontalunterricht handelt
es sich hierbei also um eine Lehrform, die sich sehr auf den Lerner als Individuum konzentriert.

\section{Adaptionsmaßnahmen}
Um eine Lernsituation adaptiv zu gestalten, werden Maßnahmen angewandt, die sich grundsätzlich in zwei
unterschiedliche Kategorien einteilen lassen. So unterscheidet man zwischen Maßnahmen auf der Makro- und Mikroebene.

Maßnahmen in der Makroebene einer Lernumgebung beschreiben Aktionen auf Klassenebene.
So werden zum Beispiel Gruppen nach Leistungsniveau eingeteilt und der Lernplan im Gesamten für
diese Einteilungen gruppenindividuell angepasst. Eine weitere Maßnahme auf der Makroebene beschreibt
die Einführung von kooperativem Lernen. So wird invididuelles Wissen der Lernenden über einen sozialen Austausch
über eine bestimmte Thematik revidiert, integriert, neu organisiert oder weiter ausdifferenziert.

Dem gegenüber stehen Maßnahmen, die eine direkte Interaktion und/oder Kommunikation zwischen Lehrer und Lernendem beschreiben.
Diese werden als Maßnahmen in der Mikroebene bezeichnet. Durch die Beschäftigung mit einem Lernenden
ist es dem Lehrer möglich, besser auf die individuellen Stärken und Schwächen eines Lernenden einzugehen.
Das Ziel ist, den Unterrichtsinhalt und die Lernmaßnahmen besser an jeden einzelen Lernenden anzupassen.

\section{Adaptionszwecke}
Betrachtet man die Frage Wie und Warum Adaptives Lernen eingesetzt wird, lassen sich drei Adaptionszwecke erkennen.
Diese stellen dar, auf welche Aspekte des Lernenden als Individuum geachtet und eingegangen wird, um die
Qualität und Effektivität des Lernens zu verbessern.
Es wird zwischen dem Fördermodell, dem Kompensationsmodell und dem Präferenzmodell unterschieden.

\subsection{Fördermodell}
Das Ziel des Fördemodells ist es, Lerndefizite des Lernenden zu beseitigen.
Meist wird dies durch eine Anpassung in Form von zusätzlichen Lerneinheiten erreicht.
Hierfür müssen die Lerndefizite beispielsweise über zusätzliche Tests oder Prüfungen erkannt
und beseitigt werden. \cite[S. 19]{lehmann2010lernstile}

\subsection{Kompensationsmodell}
Das Kompensationsmodell richtet sich auf Lernende mit unzureichenden Lernvoraussetzungen.
Diese können beispielsweise durch Überforderung oder eine generell niedrige Motivation entstehen.
Hier wird versucht, dem Lernenden durch geeignete Hilfestellung abhilfe zu leisten, um die Defizite zu kompensieren.
\cite[S. 19]{lehmann2010lernstile}

\subsection{Präferenzmodell}
Anders als in den beiden vorherigen Modellen, in denen auf die Schwächen eines Lernenden eingegangen wird,
sollen im Präferenzmodell die Stärken eines Lernenden nutzbar gemacht werden. Ist erkennbar, dass
der Lernende in einem bestimmten Bereich über besondere Lernvoraussetzungen verfügt, sollen diese genutzt werden,
um den Lernprozess für diesen Lernenden zu optimieren.
\cite[S. 19]{lehmann2010lernstile}
