

\documentclass{beamer}
\beamertemplatenavigationsymbolsempty

\usepackage[utf8]{inputenc}         % Input encoding (allow direct use of special characters like "ä")
%\usepackage[english]{babel}
\usepackage[ngerman]{babel}
\usepackage[T1]{fontenc}
\usepackage[automark]{scrpage2} 	 % Schickerer Satzspiegel mit KOMA-Script
\usepackage{setspace}           	 % Allow the modification of the space between lines
\usepackage{caption}

\captionsetup[figure]{labelformat=empty}

\usepackage{pdfpages}
% Um .eps Files einzubinden
\usepackage{epstopdf}



\setbeamercovered{transparent}
\beamertemplatenavigationsymbolsempty
\setbeamertemplate{footline}[frame number]

\usetheme{Madrid}

\AtBeginSection[]{
  \begin{frame}
  \vfill
  \centering
  \begin{beamercolorbox}[sep=8pt,center,shadow=true,rounded=true]{title}
    \usebeamerfont{title}\insertsectionhead\par%
  \end{beamercolorbox}
  \vfill
  \end{frame}
}


\title[Seminar]{Seminar e-Learning und Wissenskommunikation}
\subtitle[Remailer]{Adaptives Lernen}
\author[M. McCreight]{Mervyn McCreight}
\institute[FH-Wedel]{FH-Wedel}

\subject{Adaptives Lernen}
\keywords{Adaptives Lernen, Lernsoftware, Intelligente Tutorielle Systeme, Lernen, Lernparadigma}

\begin{document}

\frame{\titlepage}

\begin{frame}
	\frametitle{Inhaltsverzeichnis}
	\tableofcontents
\end{frame}

\section{Adaptives Lernen in der Lerntheorie}
\subsection{Vergleich zum klassischen Lehrmodell}
\subsection{Aptitude-Treatment Interaktion}
\subsection{Adaptionsmaßnahmen}
\subsection{Adaptionszwecke}


\section{Intelligente Tutorielle Systeme}
\subsection{Definition}
\subsection{Unterschied zu klassischen Lehrsystemen}
\subsection{Ablauf}
\subsection{Struktur}
\subsection{Möglichkeiten zur Umsetzung von Adaption}


\section{Beispiel}
\subsection{LISP-Tutor}
\subsection{BRIDGE-Tutor}

\section{Fazit}

\end{document}
