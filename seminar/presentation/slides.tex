

\documentclass{beamer}
\beamertemplatenavigationsymbolsempty

\usepackage[utf8]{inputenc}         % Input encoding (allow direct use of special characters like "ä")
%\usepackage[english]{babel}
\usepackage[ngerman]{babel}
\usepackage[T1]{fontenc}
\usepackage[automark]{scrpage2} 	 % Schickerer Satzspiegel mit KOMA-Script
\usepackage{setspace}           	 % Allow the modification of the space between lines
\usepackage{caption}

\captionsetup[figure]{labelformat=empty}

\usepackage{pdfpages}
% Um .eps Files einzubinden
\usepackage{epstopdf}



\setbeamercovered{transparent}
\beamertemplatenavigationsymbolsempty
\setbeamertemplate{footline}[frame number]

\usetheme{Madrid}

\AtBeginSection[]{
  \begin{frame}
  \vfill
  \centering
  \begin{beamercolorbox}[sep=8pt,center,shadow=true,rounded=true]{title}
    \usebeamerfont{title}\insertsectionhead\par%
  \end{beamercolorbox}
  \vfill
  \end{frame}
}


\title[Seminar]{Seminar e-Learning und Wissenskommunikation}
\subtitle[Remailer]{Adaptives Lernen}
\author[M. McCreight]{Mervyn McCreight}
\institute[FH-Wedel]{FH-Wedel}

\subject{Adaptives Lernen}
\keywords{Adaptives Lernen, Lernsoftware, Intelligente Tutorielle Systeme, Lernen, Lernparadigma}

\begin{document}

\frame{\titlepage}

\begin{frame}
	\frametitle{Inhaltsverzeichnis}
	\tableofcontents
\end{frame}

\section{Adaptives Lernen in der Lerntheorie}
  \begin{frame}
    \frametitle{Bedeutung}
    \begin{block}{Bedeutung}
      Adaptives Lernen bedeutet, Lernangebote für den Unterricht zu finden, die Schüler trotz unterschiedlicher Voraussetzungen, gleichermaßen fördern.
    \end{block}

    \centering
    \begin{itemize}
      \item Anpassung der Lernumgebung
      \item Dynamischer Unterricht
      \item Individualität
    \end{itemize}
  \end{frame}
\subsection{Vergleich zum klassischen Lehrmodell}
  \begin{frame}
   \frametitle{Vergleich Lernparadigmen}
   \begin{block}{Vergleich Lernparadigmen}
      \begin{table}[!htbp]
        \centering
        \begin{tabular}{c || c | c}
          \hline
          \  & \textbf{Behaviorismus} & \textbf{Kognitivismus} \\
          \hline
          \textbf{Hirn is} & passiver Behälter & Informationsverarbeitend \\
          \textbf{Wissen ist} & Input-Output Relation & interner Verarbeitungsprozess \\
          \textbf{Paradigma} & Stimulus-Response & Problemlösung  \\
          \textbf{Strategie} & Lehren & Beobachten und Helfen \\
          \textbf{Lehrer ist} & Autorität & Tutor \\
          \textbf{Interaktion} & starr & dynamisch, abhängig von Tutorand \\
        \end{tabular}
      \end{table}
   \end{block}
  \end{frame}

  \begin{frame}
  \frametitle{Vergleich Lernparadigmen}
    \begin{alertblock}{Behaviorismus}
     \begin{itemize}
       \item Alle lernen gleich
       \item statisch geplanter Unterricht
       \item Wissensreplikation
     \end{itemize}
    \end{alertblock}

    \begin{block}{Kognitivismus}
     \begin{itemize}
       \item Lernen ist individuell
       \item dynamisch angepasster Unterricht
       \item Problemlösung
     \end{itemize}
    \end{block}
  \end{frame}
\subsection{Aptitude-Treatment Interaktion}
\begin{frame}
  \frametitle{Aptitude-Treatment Interaktion}

  \begin{block}{Zweck}
    Forschung, um Nachzuweisen, dass Lernen individuell ist
  \end{block}

  \begin{block}{deutsch:}
    Fähigkeits-Verfahrens-Wechselbeziehung
  \end{block}

  \begin{itemize}
    \item Grundfähigkeiten: Charakter, Vorwissen, Lerntyp
    \item Verfahren: Lehrmethoden, Lehrmittelpräsentation
    \item Führte zur Betrachtung von adaptivem Lernen
  \end{itemize}
\end{frame}
\subsection{Adaptionsmaßnahmen}
\begin{frame}
  \frametitle{Adaptionsmaßnahmen - Makroebene}
  \begin{block}{Makroebene}
    \begin{itemize}
      \item Maßnahmen auf Klassenebene
      \item Einteilung nach Leistungsniveau
      \item Angepasster Lehrplan für die Gruppen
    \end{itemize}
  \end{block}

  Beispiel: Altes Schulsystem - Hauptschule, Realschule, Gymnasium
\end{frame}

\begin{frame}
  \frametitle{Adaptionsmaßnahmen - Mikroebene}
  \begin{block}{Mikroebene}
    \begin{itemize}
      \item direkte Kommunikation
      \item Eingehen auf Stärken und Schwächen
      \item individuelle Anpassung der Lehrmethoden
      \item laufender Anpassungsprozess des Unterrichts
    \end{itemize}
  \end{block}

  Beispiele: Verschiedene Lerntypen - bildliche oder textliche Erklärung passt besser
\end{frame}

\subsection{Adaptionszwecke}
\begin{frame}
  \frametitle{Adaptionszwecke - Fördermodell}
  \begin{block}{Fördermodell}
    \begin{itemize}
      \item Beseitigung von Lerndefiziten
      \item Verständnis möglich, Wissen noch nicht erreicht.
      \item Zusatzaufgaben
      \item Schüler fördern, bis Lernziel erreichbar ist.
    \end{itemize}
  \end{block}
\end{frame}

\begin{frame}
  \frametitle{Adaptionszwecke - Kompensationsmodell}
  \begin{block}{Kompensationsmodell}
    \begin{itemize}
      \item Kompensation von Lerndefiziten
      \item Ausgleich unzureichender Lernvoraussetzungen
      \item schlechte Motivation, Überforderung
      \item individuelle Hilfestellungen - z.B. Betreeung, Nachhilfe
    \end{itemize}
  \end{block}
\end{frame}

\begin{frame}
  \frametitle{Adaptionszwecke - Präferenzmodell}
  \begin{block}{Präferenzmodell}
    \begin{itemize}
      \item Verwendung von individuellen Stärken und Schwächen
      \item besondere Voraussetzungen ausnutzen
      \item Anpassung der Aufgaben und des Unterrichts
      \item schnellerer Lernerfolg
    \end{itemize}
  \end{block}
\end{frame}

\section{Adaptives Lernen im e-Learning}
\begin{frame}
  \frametitle{Motivation}
    \begin{alertblock}{Bisher}
      \begin{itemize}
        \item behavioristische Lernsysteme
        \item menschliche Unterstützung
        \item nicht \glqq modern\grqq{} - Lernforschung
      \end{itemize}
    \end{alertblock}

    \begin{block}{Ziel}
      \begin{itemize}
        \item aktuelle Lernforschung berücksichtigen
        \item keine menschliche Unterstützung
        \item gleichwertig mit normalem Unterricht
      \end{itemize}
    \end{block}
\end{frame}

\begin{frame}
  \frametitle{Möglichkeiten}
  \begin{block}{Hypermediale Lernsysteme}
    \begin{itemize}
      \item Verbund von hypermedialen Wissenseinheiten
      \item freie, angepasste Navigation
      \item vielfältige Präsentationsauswahl
      \item entdeckendes Lernen
    \end{itemize}
  \end{block}

  \begin{block}{Intelligente Tutorielle Systeme}
    \begin{itemize}
      \item Erweiterung klassischer Lernsoftware
      \item Lehrverhalten angepasst an Lerner
      \item Tutor = Unterstützer
    \end{itemize}
  \end{block}
\end{frame}


\subsection{Intelligente Tutorielle Systeme}
\begin{frame}
  \frametitle{Intelligente Tutorielle Systeme}
  \begin{block}{Definition}
    Intelligente tutorielle Systeme (ITS) sind adaptive Mediensysteme, die sich ähnlich
    einem menschlichen Tutor an die kognitiven Prozesse des Lernenden anpassen
    sollen, indem sie die Lernfortschritte und -defizite analysieren und dementsprechend
    das Lernangebot generativ modifizieren sollen.
  \end{block}

  \begin{itemize}
    \item Adaptivität
    \item Adaptierbarkeit
  \end{itemize}
\end{frame}

\begin{frame}
  \frametitle{Grundanforderungen}
  \begin{block}{Adaptivität}
    \begin{itemize}
      \item Lehrplan und Geschwindigkeit, Aufgabentyp
      \item dynamisch während des Lernens
      \item System muss mit Lernen --> Lerner
    \end{itemize}
  \end{block}

  \begin{block}{Flexibilität}
    \begin{itemize}
      \item Darstellung Lerninhalte
      \item angepasst an Lerner
    \end{itemize}
  \end{block}

  \begin{block}{Diagnosefähigkeit}
    \begin{itemize}
      \item Kernaspekt
      \item Analyse des Lernenden
      \item Wissensstand
      \item Stereotyp
    \end{itemize}
  \end{block}
\end{frame}
\subsection{Unterschied zu klassischen Lehrsystemen}
\subsection{Ablauf}
\subsection{Struktur}
\subsection{Möglichkeiten zur Umsetzung von Adaption}


\section{Beispiel}
\subsection{LISP-Tutor}
\subsection{BRIDGE-Tutor}

\section{Fazit}

\end{document}
